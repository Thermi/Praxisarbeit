\section{High Availability}


High Availability is usually implemented by duplicating the service over several distinct physical or
virtual hosts. The hosts are called a \"cluster\". The software on the hosts and other resources
that are duplicated and necessary for the service are managed by what is called a \"cluster manager\".
Said manager handles cluster communication and failover action over several hosts if a local service,
that is protected by HA\footnote{High Availability} goes down.

\section{Cluster Types}

There are diverse cluster types for different purposes.
\begin{itemize}
\item High Availability Cluster to increase the availability of services.
\item High Performance Clusters to increase the performance of services.
\item Load Balancing Cluster to balance the load of a service.
\end{itemize}

In this work, we will only look at High Availability Clusters.
