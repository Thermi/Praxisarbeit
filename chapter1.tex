% This file is part of Praxisarbeit

% Praxisarbeit is free software: you can redistribute it and/or modify
% it under the terms of the GNU General Public License version 3 as published by
% the Free Software Foundation.

% Praxisarbeit is distributed in the hope that it will be useful,
% but WITHOUT ANY WARRANTY; without even the implied warranty of
% MERCHANTABILITY or FITNESS FOR A PARTICULAR PURPOSE.  See the
% GNU General Public License for more details.

% You should have received a copy of the GNU General Public License
% along with Foobar.  If not, see <http://www.gnu.org/licenses/>.
\section{High Availability}


High Availability is usually implemented by duplicating the service over several distinct physical or
virtual hosts. The hosts are called a \"cluster\". The software on the hosts and other resources
that are duplicated and necessary for the service are managed by what is called a \"cluster manager\".
Said manager handles cluster communication and failover action over several hosts if a local service,
that is protected by \ac{HA} goes down.

\section{Cluster Types}

Naturally, there exist different types of clusters for different purposes.
\begin{itemize}
\item High Availability Cluster to increase the availability of services.
\item High Performance Clusters to increase the performance of services.
\item Load Balancing Cluster to balance the load of a service.
\end{itemize}

In this work, we will only look at High Availability Clusters.
High Availability Clusters are usually used for critical infrastructure,
which must be available most of the time. Examples for such a case are DNS servers,
routers and firewalls, as well as \ac{ERP} systems.
