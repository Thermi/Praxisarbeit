\section{Setting up a HA cluster on Linux}
This part is about the real wordl installation of an \ac{HA} cluster on Linux using the tools introduced in the former sections.

\subsection{Cluster architecture}
For the cluster, we will use 3 CentOS 7 \acp{VM}, which are connected
to three \acp{VLAN}: 
\begin{itemize}
\item 192.168.178.0/24 for public cluster resources
\item 192.168.180.0/24 for the management of the clusters
\item 192.168.181.64/27 for the cluster communication
\end{itemize}
The services we are going to set up are the following:
\begin{itemize}
\item A cluster IP
\item A lighttpd web server to host an example web site
\item A shared storage built with DRBD and GFS2
\end{itemize}

Because the operating system the hypervisor is running on does
not offer the necessary software to implement fencing, so this setup will not use it. 
However, it is greatly advised to use fencing.

% Test
% Performance
