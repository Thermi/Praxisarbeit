
\section{Data Storage}

The hosts in the cluster usually provide a service for which persistent shared data storage is necessary.
If those files are static or are rarely modified, locally storing them on each node can be a solution. However, caution is advised, as makeshifts have a bad habit of sticking around.
There are two different feasible ways to implement this in a cluster environment:
\begin{itemize}
\item Using a storage area network
\item Using a network access storage
\item using shared storage on the cluster nodes
\end{itemize}
% TODO: Graphic of the different layers of file storage in a cluster
Using a storage area network in a cluster environment is good choice as it provides highly available access, if the SAN\footnote{storage area network} is layed out for high availability. That is the case in enterprise environments. As such, it can be relied upon.
Using network access storage is a more common solution for providing access to data in an environment where concurrent access is needed, but cluster file systems are not an option. In such a case, the cluster members would access the data over network protocols like NFS\footnote{Network File System} or CIFS\footnote{Common Internet File System}. For concurrent access to shared storagein a cluster, file systems like GFS\footnote{Gluster File System}, OCFS2\footnote{Oracle Cluster File System 2} or CephFS have to be used, as those provide a lock manager, which is needed for data integrity. In the following sections, we will go over some cluster file systems and shared storage providers, as well as Ceph in an extra section.
\subsection{Storage Area Network}
\begin{bytefield}[boxformatting={\centering\itshape},
bitwidth=.8em,
endianness=big]{32}
\bitbox{32}{files} \\
\bitbox{32}{Cluster File System} \\
\bitbox{32}{Block Device on the cluster member} \\
\bitbox{32}{Network Protocol (iSCSI, ...)} \\
\bitbox{32}{Storage Area Network}\\
\end{bytefield}

% TODO: More information about cluster file systems and their exact characteristica (behaviour during failure, performance, scalability), also: more shared storage providers
Storage Area Networks are a common concept in enterprises and are mostly implemented with appliances from different vendors, which cluster and provide high availability. The nodes in need of shared storage import the shared storage with iSCSI or other network protocols that the own host's kernel can display as block devices. Over that block device, a cluster file system must be layed over the shared medium to handle concurrent access.
Obviously, SPoFs in the connection to the SAN must be avoided, for which different solutions can be used.
\subsection{Network Access Storage}
\begin{bytefield}[boxformatting={\centering\itshape},
bitwidth=.8em,
endianness=big]{32}
\bitbox{32}{files} \\
\bitbox{32}{Network Protocol (NFS, CIFS, ...)} \\
\bitbox{32}{Network Access Storage} \\
\end{bytefield}
NAS are basicly network hard drives, which provide file storage to other hosts on the network. They do not provide distributed storage like a SAN and are usually not layed out redundantly. NAS export their data using network file systems like NFS or CIFS, which provide their own lock manager. This solves concurrency issues. Some NAS are also capable of using other network file systems like iSCSI to provide access to data on the block layer, which enables the use of cluster file systems on top of those block devices.
\subsubsection{NFS}
NFS\footnote{Network File System} is a distributed file system protocol, which is deployed widely on Linux and UNIX. 
The latest version is 4.1 and is openly standardized in RFCs, so anyone can implement it.
%TODO: Protocols, performance, set up
%TODO: Write down all the RFCs for it
\paragraph{protocols}
NFS can be used over TCP and UDP.
The TCP protocol is stateful and the UDP protocol is stateless.
\paragraph{performance}
\paragraph{lock manager}
NFS has no intrinsic lock manager in the protocol. However, there is the NLM\footnote{Network Lock Manager}
protocol, which can work together with the normal NFS protocol to provide a lock manager.
It is a seperate service of nfs.
\linebreak[3]
NFS currently does not provide a distributed lock manager, therefore a breakdown
of communication betweenh the nfs client and the nfs server will result in stale locks
on the server until the connection times out. Also, there is no possibility
for the deployment of reundant servers. As such, the loss of a single server prevents
access to the shared resources.

\subsubsection{CIFS}
\paragraph{protocols}
\paragraph{performance}
\paragraph{lock manager}
\subsection{Shared Storage On The Cluster Nodes}
\begin{bytefield}[boxformatting={\centering\itshape},
bitwidth=.8em,
endianness=big]{32}
\bitbox{32}{files} \\
\bitbox{32}{Cluster File System} \\
\bitbox{32}{Shared Storage Implementation of your choice} \\
\bitbox{32}{Block Device on the cluster member} \\
\bitbox{32}{Specific Implementation of Shared Storage} \\
\end{bytefield}
Shared storage on the cluster nodes is a similiar topic to storage are network. The difference is,
that the block devices are stored locally on each cluster node and not in a Storage Area Network. In such a scenario, data replication has to be taken care of by the system's kernel or another application. A solution for this on Linux is DRBD, which replicates data over the network using TCP.
\paragraph{DRBD}
DRBD stands for Dynamic Redundant Block Devices and is an open source technology developed by LINBIT.
It provides data replication over a cluster of two hosts. %TODO: Describe how it works
As of the time of the writing, DRBD only works good for two nodes, but scales bad beyond it. In the next version of DRBD, support vor arbitrary numbers of cluster members will be added, which enables cheap shared storage for multiple cluster members.
\begin{bytefield}[boxformatting={\centering\itshape},
bitwidth=.8em,
endianness=big]{32}
\bitbox{32}{files} \\
\bitbox{32}{Cluster File System} \\
\bitbox{32}{Block Device on the cluster member} \\
\bitbox{32}{DRBD block device on the cluster member} \\
\bitbox{32}{DRBD metadata and data layer} \\
\bitbox{32}{Hard drive partition} \\
\bitbox{32}{Kernel} \\
\bitbox{32}{$\cdots$} \\
\end{bytefield}
\subsubsection{Cluster File Systems}
\paragraph{GFS2}
GFS2\footnote{Global File System 2} is the second iteration of a cluster file system of Redhat, Inc. It provides different lock managers:
\begin{description}
\item distributed lock manager 
\item nolock manager
\end{description}
\paragraph{GlusterFS}
\paragraph{OCFS2}
OCFS2\footnote{Oracle Cluster File System 2} is a cluster file system developed by Oracle.
Its original purpose was to serve as cluster file system for Oracle's database software.
It provides POSIX semantics and is part of the Linux kernel since kernel version 2.6.16. 
%TODO: Complete
\linebreak[3]
Its current version is 1.6. Support for it is very limited, because Oracle only
provides any for it as part of Oracle Linux. SUSE provides support, too, but only
if it is used on Novell's SUSE Linux Exterprise Server. The tool chain to manage
OCFS2 file systems is available on Oracle's web site\footnote{\url{https://oss.oracle.com/projects/ocfs2-tools/}},
however the tool chain needs patches to compile correctly. Therefore, it is advised
to look at the rpm source of the working packages for CentOS or other distributions and use those as a basis
to get a working tool chain.
\paragraph{cLVM2}
cLVM2 is the clustered version of LVM2, which is a volume manager for Linux.
The clustered version enables the synchronization of the status and metadata
of the cLVM volumes over the network.

%TODO: Complete
\paragraph{Performance Considerations}
Usually, for High Availability clusters, it is sufficient for a cluster file system
to have at least some performance. In numbers, this means the processing of 
hundres of transactions, not thousands or more. A tradeoff of performance for
availability is usually desireable.

\paragraph{High Availability Considerations}
For high availability, it is necessary to have a distributed lock manager to prevent
a loss of availability if the host which currently has the lock manager, goes down or
is unreachable for any reason.
\subsubsection{Shared Storage Providers}
\subsubsection{Ceph}
Ceph has a special position in the cluster world, as it provides all aspects of shared storage:
\begin{itemize}
\item block devices
\item cluster file systems
\item object storage
\end{itemize}

Ceph is an open source cluster storage platform to provide block, file and object storage to other hosts. It is constructed in a manner that prevents SPoFs\footnote{Single Point Of Failure}, scales well to exabyte levels and provides redundancy to prevent data loss. It is capable of distributing the data across the cluster members and maintaining performance over an unlimited number of cluster members.

Ceph can be used as shared storage provider, over which an arbitrary cluster file system can be layed, but also completely with CephFS as complete shared storage solution. Furthermore, it can be used as storage backend for libvirt and qemu to store virtual machines in. It also provides a gateway for object storage over a REST API and is compliant with the APIs provided by Amazon S3 and Swift.% TODO: Cite!

Users authenticate themselves to the cluster using a mechanism that utilises shared keys and provides authenticity to prevent MITM\footnote{Man In The Middle} attacks. However, no secrecy or replay protection is provided\cite{ceph_architecture}.

Ceph is capable of distributing copies of stored objects over the cluster, as well as striping and dynamically moving the objects to balance the load on the different hosts in the storage cluster. A such, it is a perfect solution for large storage clusters.

\section{Fencing/STONITH}
\subsection{Purpose}
Fencing and STONITH basicly describe the same thing: The prevention of I/O-Operations of a disconnected/misbehaving host in the cluster. For this purpose, two difference solutions have been developed:
\begin{description}
\item[Fencing] foo
\item[STONITH] bar
\end{description}
\subsection{Fencing}
\subsection{Stonith}
\begin{description}
\item[S]hoot
\item[T]he
\item[O]ther
\item[N]ode
\item[I]n
\item[T]he 
\item[H]ead
\end{description}
