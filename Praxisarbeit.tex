%
% File Name     : Praxisarbeit.text
% Purpose       :
% Creation Date : 2015-05-25
%
\documentclass[a4paper]{scrartcl}
\usepackage{listing}
\usepackage{amsfonts}
\usepackage{amssymb}
\usepackage{listings}
\usepackage{tabularx}
\usepackage[utf8]{inputenc}
\usepackage[english]{babel}

\usepackage{amsmath}
\usepackage{graphicx}
\usepackage[hidelinks]{hyperref}
\usepackage{pdfpages}
\usepackage{pgfplots}
%\usepackage{biblatex}
\usepackage{bytefield}
\usepackage[printonlyused]{acronym}
\usepackage{setspace}

\pgfplotsset{compat=1.5}
\newcommand*\BlankPage{\newpage\null\thispagestyle{empty}\newpage}
\setstretch{1.3}
\newcommand{\comment}[1]{}  %comment not shown
\lstset{
breaklines=true,
literate={ö}{{\"o}}1
         {ä}{{\"a}}1
         {ü}{{\"u}}1
         {ß}{{\ss}}2
         {Ä}{{\"A}}1
         {Ü}{{\"U}}1
         {Ö}{{\"O}}1
}

\begin{document}
\pagenumbering{roman}
\title{Praxisarbeit - Clustering with Linux in the year 2015}
\author{Noel Kuntze }
\maketitle

\BlankPage

\phantomsection
\addcontentsline{toc}{section}{Abstract}

\section*{Abstract}
As Linux is the most widely spread operating system in the data center environment,
implementing high availability for services on that platform is an important aspect to
critical network services. This paper lists and asesses the modern implementations
of software for high availability on the Linux platform.

\cleardoublepage
\phantomsection

\addcontentsline{toc}{section}{Abbildungsverzeichnis}
\listoffigures

\cleardoublepage
\section*{Abkürzungsverzeichnis}
\addcontentsline{toc}{section}{Abkürzungsverzeichnis}
\begin{acronym}
% Akronyme
    \acro{API}{Application Programming Interface}
    \acro{ABI}{Application Binary Interface}
    \acro{BMC}{Base Management Controller}
    \acrodefplural{BMC}[BMCs]{Base Management Controllers}
    \acro{CIB}{Cluster Information Base}
    \acro{CIFS}{Common Internet File System}
    \acro{CRMd}{Cluster Resource Management daemon}
    \acro{cLVM2}{Clustered Logical Volume Manager 2}
    \acro{DRBD}{Dynamic Redundant Block Devices}
    \acro{ERP}{Enterprise Resource Planning}
    \acro{GFS}{Global File System}
    \acro{GFS2}{Global File System 2}
    \acro{GlusterFS}{Gluster File System}
    \acro{HA}{High Availability}
    \acro{LRMd}{Local Resource Management daemon}
    \acro{MITM}{Man In The Middle}
    \acro{NAS}{Network Access Storage}
    \acro{NFS}{Network File System}
    \acro{NLM}{Network Lock Manager}
    \acro{OCFS}{Oracle Cluster File System}
    \acro{OCFS2}{Oracle Cluster File System 2}
    \acro{PE}{Policy Engine}
    \acro{RPC}{Remote Procedure Call}
    \acro{SAN}{Storage Area Network}
    \acro{SMB}{Server Message Block}
    \acro{STONITHd}{Fencing daemon}
    \acro{STONITH}{Shoot The Other Node In The Head}

    \acro{VLAN}{Virtual Local Area Network}
    \acrodefplural{VLAN}[VLANs]{Virtual Local Area Networks}
    
    \acro{VM}{Virtual Machine}
    \acrodefplural{VM}[VMs]{Virtual Machines}
    
\end{acronym}

\newpage

\tableofcontents
\setcounter{tocdepth}{2}

\newpage
\pagenumbering{arabic}

\section{Introduction}

High availability is an important aspect of network services in the corporate
sector, as it decreases the time a service is unavailable to users.
Such services might be business critical, like ERP\footnote{Enterprise Resource Planning} systems.
Therefore it is necessary to evaluate and implement high availability in those systems.

\subsection{The scope of this document}
The purpose of this document is to explain and promote the modern implementations 
of high availability clustering on Linux, as well as judge the current state
of them and availability on different Linux distributions.
For this purpose, different tools for this purpose will be explained in 
depth with their features and services to the cluster environment. 
Furthermore, this document contains guidance on optimizing cluster performance, 
failure detection and guidance on configuring a high availability cluster 
scenario on three CentOS 7 hosts in a virtualized environment.


\section{High Availability}


High Availability is usually implemented by duplicating the service over several distinct physical or
virtual hosts. The hosts are called a \"cluster\". The software on the hosts and other resources
that are duplicated and necessary for the service are managed by what is called a \"cluster manager\".
Said manager handles cluster communication and failover action over several hosts if a local service,
that is protected by \ac{HA} goes down.

\section{Cluster Types}

Naturally, there exist different types of clusters for different purposes.
\begin{itemize}
\item High Availability Cluster to increase the availability of services.
\item High Performance Clusters to increase the performance of services.
\item Load Balancing Cluster to balance the load of a service.
\end{itemize}

In this work, we will only look at High Availability Clusters.
High Availability Clusters are usually used for critical infrastructure,
which must be available most of the time. Examples for such a case are DNS servers,
routers and firewalls, as well as \ac{ERP} systems.



\section{Data Storage}

The hosts in the cluster usually provide a service for which persistent shared data storage is necessary.
If those files are static or are rarely modified, locally storing them on each node can be a solution. However, caution is advised, as makeshifts have a bad habit of sticking around.
There are two different feasible ways to implement this in a cluster environment:
\begin{itemize}
\item Using a storage area network
\item Using a network access storage
\end{itemize}

Using a storage area network in a cluster environment is good choice as it provides highly available access, if the SAN\footnote{storage area network} is layed out for high availability. That is the case in enterprise environments. As such, it can be relied upon.
Using network access storage is a more common solution for providing access to data in an environment where concurrent access is needed, but cluster file systems are not an option. In such a case, the cluster members would access the data over network protocols like NFS\footnote{Network File System} or CIFS\footnote{Common Internet File System}. For concurrent access to shared storagein a cluster, file systems like GFS\footnote{Gluster File System}, OCFS2\footnote{Oracle Cluster File System 2} or CephFS have to be used, as those provide a lock manager, which is needed for data integrity. In the following sections, we will go over some cluster file systems and shared storage providers, as well as Ceph in an extra section.
\subsection{Storage Area Network}
% TODO: More information about cluster file systems and their exact characteristica (behaviour during failure, performance, scalability), also: more shared storage providers
\subsubsection{Cluster File Systems}
\paragraph{GFS2}
\paragraph{GlusterFS}
\paragraph{OCFS2}
\subsubsection{Shared Storage Providers}
\paragraph{DRBD}
\subsubsection{Ceph}
Ceph has a special position in the cluster world, as it provides all aspects of shared storage:
\begin{itemize}
\item block devices
\item cluster file systems
\item object storage via an API
\end{itemize}

Ceph can be used as shared storage provider, over which an arbitrary cluster file system can be layed, but also completely with CephFS as complete shared storage solution. Furthermore, it can be used as storage backend for libvirt and qemu to store virtual machines in.

\subsection{Network Access Storage}

\section{Software}
To implement High Availability, different types of software for different purposes are used.
As mentioned before, a cluster manager is used for communication between the different hosts that
provide services in the cluster. Services are commonly called "resources" in the cluster world.
Another software is the resource manager, which manages the resources themselves.
This includes starting, stopping and monitoring the resources.
\subsection{Pacemaker}
Pacemaker is a ressource manager for \ac{HA} clusters. Every node in a cluster
runs an instance of it. It detects failures of system daemons and the physical or
virtual host itself. Also, it handles the complex relationship between the different
services and resources, which are formulated by the administrator.
Pacemaker uses the communication facilities, which are provided by Corosync
to communicate with the Pacemaker instances on the other nodes. It is primarily written in C.
\subsubsection{History}
Pacemaker is primarily a joint effort between Red Hat and Novell, but also
some other, smaller companies and the open source community. The project
exists since 2004.
\subsubsection{Components}
\begin{itemize}
\item \ac{CIB}
\item \ac{CRMd}
\item \ac{LRMd}
\item \ac{PE}
\item \ac{STONITHd}
\end{itemize}
%TODO: Service structure diagram
\subsection{Corosync}
Corosync is a descendent of the older OpenAIS project and is run as a system daemon. The software is open source and licensed under the BSD licenses. It provides group membership communication and enables the cluster members to communicate with each other on the application layer to communicate service outages, corruption and scheduled shutdowns of hosts other hosts. The daemon is written in C and is commonly used on the Linux platform together with pacemaker or cman.

The latest version of Corosync as of the time of the writing was $2.3.5$.
%TODO: Service structure diagram
\subsubsection{History}
\subsubsection{Services}
Corosync takes up the whole task of managing the cluster and distributing the services. For this purpose, it provides 4 different service engines that can be used with the native C \ac{API}:
\begin{description}
\item [confdb] Configuration and Statistics database
\item [cpq] Closed Process Group
\item [quorum] Provides notifications of gain or loss of quorum
\item [sam] Simple Availability Manager
\end{description}
The native C API of Corosync uses shared memory for high throughput and low 
latency and enables the efficient use of the API for mass sending of messages to 
other cluster members.
\linebreak[3]
Corosync can use \ac{UDP} over IP or Infiniband for communication and supports different 
rings of communication for redundancy and prevention of false positives in the 
detection of unresponsive or dead hosts.
\subsubsection{Locality Constraints}
Corosync can keep services together on a single host and express locality constraints in general,
like pinning services to a certain host or bundling services together, which should
always be together, like a cluster IP and a web service.
\subsubsection{Failover}
Corosync uses totems\cite{Amir95thetotem} to check the availability of the hosts
in the cluster. It can use different ''rings''\footnote{Different communication
paths between the hosts, e.g: physical networks} to detect a failure
in the network infrastructure between the host, but not actually causing any failover
action on the host themselves.

\subsection{pcs}
\ac{pcs} is a modern program to configure and maintain clusters made of pacemaker and Corosync. It provides a unified command line interface to the configuration options of Corosync and pacemaker, which enables the easy configuration of the aforementioned software. pcs manages
the hosts using a daemon called ''pcsd'', to which the other nodes authenticate using a Unix account.
The daemon runs as root user and manages the \ac{CIB} and the status of pacemaker and Corosync on the host.
When setting up a cluster using pcs, it does not generate an authkey file by default to secure the
cluster traffic. This must be done to the user, because it takes several minutes.
The upstream url is \url{https://github.com/feist/pcs}. The software is primarily maintained by Chris Feist of Red Hat.
% pcsd
% Port 2224 TCP
% unix account auth
% no authkey by default

\subsection{The whole picture}
\subsection{Availability of the software in different Linux distributions}
% Debian https://packages.debian.org/stable/allpackages?format=txt.gz
% CentOS
% RHEL
% Fedora
% 
\begin{table}
\begin{tabular}[!h]{|c|c|c|c|c|c|}
\firsthline
% TODO: crm shell
Distribution & pcs & Pacemaker & Corosync & fence-virt & fence-agents\\
\hline
Debian 7 & No & 1.1.7 & 1.4.2 & No & 3.1.5 \\
Debian 8 & No & No & 1.4.6 & No & 3.1.5 \\
\hline
Fedora 22 & 0.9.139 & 1.1.12 & 2.3.5 & 0.3.2 & 4.0.16 \\
Fedora 23 & 0.9.141 & 1.1.12 & 2.3.5 & 0.3.2 & 4.0.16 \\
\hline
CentOS 6 & 0.9.123 & 1.1.12 & 1.4.7 & 0.2.3 & 3.1.5 \\
CentOS 7 & 0.9.137 & 1.1.12 & 2.3.4 & 0.3.2 & 4.0.11 \\
\hline
SLES 11 HA Extension & No & Yes, version unknown & Unknown & Unknown & Unknown \\
SLES 12 HA Extension & No & Yes, version unknown & Unknown & Unknown & Unknown \\
\hline
Arch Linux & No & Yes & Yes & No & Yes \\
\lasthline
\end{tabular}
\caption{Software Availability}
\label{table:SoftwareAvailability}
\subsubsection{Plausibility of building clusters on different distros}
The content of \autoref{table:SoftwareAvailability} clearly shows, that different
distros have different support for clustering. For example, Debian has basically no support
for clustering, whereas all distros that are affiliated with Red Hat have very good
support for it. SLES is a different topic. I could not find out what software exactly
is used in the different versions, only that clustering is supported. As 
S.U.S.E Linux Gmbh, the developer of SLES, invested money around the year 2000
in the development of clustering software, it is plausible that, as usual,
pacemaker and Corosync is used, probably with crm shell.
Building clusters on Debianoid systems is not possible or advised. This
includes Ubuntu and Linux Mint.
\end{table}


\section{Setting up a HA cluster on Linux}
This part is about the real wordl installation of an HA cluster on Linux using the tools introduced in the former sections.

% VMs
% Services
% Stonith
% Test
% Performance


% This file is part of Praxisarbeit

% Praxisarbeit is free software: you can redistribute it and/or modify
% it under the terms of the GNU General Public License version 3 as published by
% the Free Software Foundation.

% Praxisarbeit is distributed in the hope that it will be useful,
% but WITHOUT ANY WARRANTY; without even the implied warranty of
% MERCHANTABILITY or FITNESS FOR A PARTICULAR PURPOSE.  See the
% GNU General Public License for more details.

% You should have received a copy of the GNU General Public License
% along with Foobar.  If not, see <http://www.gnu.org/licenses/>.
\section{Maintaining a Cluster}
Maintaining a cluster is a completely different story from setting one up.
\paragraph{maintenance}
While doing maintenance, the node should be put into standby mode, so it is not suspect
to fencing action. A fencing action can occur when the maintenance process
takes up too much CPU time for Corosync to work correctly.
When a node is put into standby mode, all resources are moved away from it.
The syntax for pcs is ''pcs cluster standby <node>''. Putting nodes
into production is done using ''pcs cluster unstandby <node>''.
\paragraph{adding a node}
From time to time, a cluster needs to be expanded.
This is done with ''pcs cluster node add <node>''.
Everything besides the pure Corosync conf needs to be moved to the new
node manually. This includes the authkey.
\paragraph{removing a node}
If a node needs to be removed, use ''pcs cluster node remove <node>''.


\section{Conclusion}
Based on my last experiences, clustering on Linux has adulterated in the 
last 15 years, at least for small clusters. The complexity is not
overwhelming and the concepts are easy to grasp. The configuration is made
easier by using ''pcs'', which eases many tasks. It still lacks support
for clustering of resources in different geographical locations like colocations.
This can be implemented using ''booth''\footnote{\url{https://github.com/ClusterLabs/booth}}.
Easy and cheap replication of data over different nodes is not available, but should
be in the near future through \ac{DRBD}. The current lack of cheap replication
makes building a cluster with shared storage, but without a SAN, not suitable
for production. Building clusters is only supported on platforms provided
by Red Hat (RHEL, CentOS, Fedora, ...) or SUSE (openSUSE, SLES).
All in all, clustering on Linux is suitable for businesses now, even without
paying experts for integration.



\newpage


\nocite{*}
\phantomsection
\addcontentsline{toc}{section}{Literature}
\bibliographystyle{alpha}
\bibliography{bibliography.bib}

\newpage
\begin{appendix}

\section{Attachement}
\end{appendix}

\end{document}
