\section{Introduction}

High availability is an important aspect of network services in the corporate
sectore, as it decreases the time a service is unavailable to users.
Such services might be business critical, like ERP\footnote{Enterprise Resource Planning} systems.
Therefore it is necessary to evaluate and implement high availability in those systems.

\subsection{The scope of this document}
The purpose of this document is to explain and promote the modern implementations 
of high availability clustering on Linux, as well as judge the current state
of them and availability on different Linux distributions.
For this purpose, different tools for this purpose will be explained in 
depth with their features and services to the cluster environment. 
Furthermore, this document contains guidance on optimizing cluster performance, 
failure detection and guidance on configuring a high availability cluster 
scenario on three CentOS 7 hosts in a virtualized environment.
