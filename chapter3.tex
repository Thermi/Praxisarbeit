\section{Software}
To implement High Availability, different types of software for different purposes are used.
As mentioned before, a cluster manager is used for communication between the different hosts that
provide services in the cluster. Services are commonly called "resources" in the cluster world.
Another software is the resource manager, which manages the resources themselves.
This includes starting, stopping and monitoring the resources.
\subsection{Pacemaker}
Pacemaker is a ressource manager for \ac{HA} clusters. Every node in a cluster
runs an instance of it. It detects failures of system daemons and the physical or
virtual host itself. Also, it handles the complex relationship between the different
services and resources, which are formulated by the administrator.
Pacemaker uses the communication facilities, which are provided by Corosync
to communicate with the Pacemaker instances on the other nodes. It is primarily written in C.
\subsubsection{History}
Pacemaker is primarily a joint effort between Red Hat and Novell, but also
some other, smaller companies and the open source community. The project
exists since 2004.
\subsubsection{Components}
\begin{itemize}
\item \ac{CIB}
\item \ac{CRMd}
\item \ac{LRMd}
\item \ac{PE}
\item \ac{STONITHd}
\end{itemize}
%TODO: Service structure diagram
\subsection{Corosync}
Corosync is a descendent of the older OpenAIS project and is run as a system daemon. The software is open source and licensed under the BSD licenses. It provides group membership communication and enables the cluster members to communicate with each other on the application layer to communicate service outages, corruption and scheduled shutdowns of hosts other hosts. The daemon is written in C and is commonly used on the Linux platform together with pacemaker or cman.

The latest version of corosync as of the time of the writing was $2.3.5$.
%TODO: Service structure diagram
\subsubsection{History}
\subsubsection{Services}
Corosync takes up the whole task of managing the cluster and distributing the services. For this purpose, it provides 4 different service engines that can be used with the native C \ac{API}:
\begin{description}
\item [confdb] Configuration and Statistics database
\item [cpq] Closed Process Group
\item [quorum] Provides notifications of gain or loss of quorum
\item [sam] Simple Availability Manager
\end{description}
The native C API of corosync uses shared memory for high throughput and low 
latency and enables the efficient use of the API for mass sending of messages to 
other cluster members.
\linebreak[3]
Corosync can use UDP over IP or Infiniband for communication and supports different 
rings of communication for redundancy and prevention of false positives in the 
detection of unresponsive or dead hosts.
\subsubsection{Locality Constraints}
Corosync can keep services together on a single host and express locality constraints in general,
like pinning services to a certain host or bundling services together, which should
always be together, like a cluster IP and a web service.
\subsubsection{Failover}
Corosync uses totems\cite{Amir95thetotem} to check the availability of the hosts
in the cluster. It can use different ''rings''\footnote{Different communication
paths between the hosts, e.g: physical networks} to detect a failure
in the network infrastructure between the host, but not actually causing any failover
action on the host themselves.
\paragraph{Tuning the fail detection}
% Note towards virtual machines
% Note towards Multicast groups and IGMP snooping
\subsection{pcs}
pcs\footnote{pacemaker corosync shell} is a modern program to configure and maintain clusters made of pacemaker and corosync. It provides a unified command line interface to the configuration options of corosync and pacemaker, which enables the easy configuration of the aforementioned software.
The upstram url is \url{https://github.com/feist/pcs}. The software is primarily maintained by Chris Feist of Redhat.
\subsection{The whole picture}
\subsection{Availability of the software in different Linux distributions}
% Debian https://packages.debian.org/stable/allpackages?format=txt.gz
% CentOS
% RHEL
% Fedora
% 
\begin{table}
\begin{tabular}{|c|c|c|c|c|c|}
\firsthline
% TODO: 
Distribution & pcs & Pacemaker & Corosync & fence-virt & fence-agents\\
\hline
Debian 7 & No & 1.1.7 & 1.4.2 & No & 3.1.5 \\
Debian 8 & No & No & 1.4.6 & No & 3.1.5 \\
\hline
Fedora 22 & 0.9.139 & 1.1.12 & 2.3.5 & 0.3.2 & 4.0.16 \\
Fedora 23 & 0.9.141 & 1.1.12 & 2.3.5 & 0.3.2 & 4.0.16 \\
\hline
CentOS 6 & 0.9.123 & 1.1.12 & 1.4.7 & 0.2.3 & 3.1.5 \\
CentOS 7 & 0.9.137 & 1.1.12 & 2.3.4 & 0.3.2 & 4.0.11 \\
\hline
SLES 11 HA Extension & Unknown & Yes, version unknown & Unknown & Unknowno & Unknown \\
SLES 12 HA Extension & Unknown & Yes, version unknown & Unknown & Unknown & Unknown \\
\hline
Arch Linux & No & Yes & Yes & No & Yes \\
\lasthline
\end{tabular}
\caption{Software Availability}
\label{table:SoftwareAvailability}
When Pacemaker is available, then the fence agents are usually available, too.
Therefore, there is no additional column for the fence agents.
\end{table}
