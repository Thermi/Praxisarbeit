\section{Running clusters on \acp{VM}}
\subsection{Why it is a bad idea}
Running a cluster on \acp{VM} is neither encouraged, nor widely supported.
The reason herein is, that the added complexity of the hypervisor decrease the availability
of the service. Because of the nature of the token protocol used by corosync,
Corosync is scheduled as a realtime process. This is necessary to ensure the quick
processing of the tocken messages. If that is not ensured, other nodes
in the cluster could believe the node is dead and will fence it to ensure the safety
and consistency of the shared storage.
As the processing capability and network is shared with other \acp{VM},
overprovisioning of the host \textbf{will} cause problems with the cluster.
The cluster nodes must be guaranteed to be able to run corosync in time.
In rare cases, the timeout values for the token protocol have to be increased,
because the hypervisor does not schedule the guest for execution early enough.
This will case corosync to log warnings over syslog:
\begin{lstlisting}
Aug 04 02:10:10 c7-arch-mirror-1 corosync[1058]: [MAIN  ] Corosync main process was not scheduled for 1217.9677 ms (threshold is 800.0000 ms). Consider token timeout increase.
\end{lstlisting}
