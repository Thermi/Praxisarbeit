\section{maintaining a cluster}
Maintaining a cluster is a completely different story from setting one up.
\paragraph{maintenance}
While doing maintenance, the node should be put into standby mode, so it is not suspect
to fencing action. A fencing action can occur when the maintenance process
takes up too much CPU time for corosync to work correctly.
When a node is put into standby mode, all resources are moved away from it.
The syntax for pcs is ''pcs cluster standby <node>''. Putting nodes
into production is done using ''pcs cluster unstandby <node>''.
\paragraph{adding a node}
From time to time, a cluster needs to be expanded.
This is done with ''pcs cluster node add <node>''.
Everything besides the pure corosync conf needs to be moved to the new
node manually. This includes the authkey.
\paragraph{removing a node}
If a node needs to be removed, use ''pcs cluster node remove <node>''.
