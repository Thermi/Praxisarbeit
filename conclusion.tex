\section{Conclusion}
Based on my last experiences, clustering on Linux has adulterated in the 
last 15 years, at least for small clusters. The complexity is not
overwhelming and the concepts are easy to grasp. The configuration is made
easier by using ''pcs'', which eases many tasks. It still lacks support
for clustering of resources in different geographical locations like colocations.
This can be implemented using ''booth''\footnote{\url{https://github.com/ClusterLabs/booth}}.
Easy and cheap replication of data over different nodes is not available, but should
be in the near future through \ac{DRBD}. The current lack of cheap replication
makes building a cluster with shared storage, but without a SAN, not suitable
for production. Building clusters is only supported on platforms provided
by Red Hat (RHEL, CentOS, Fedora, ...) or SUSE (openSUSE, SLES).
All in all, clustering on Linux is suitable for businesses now, even without
paying experts for integration.
